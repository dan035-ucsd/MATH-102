\documentclass[12pt, letterpaper]{article} 

\usepackage{amsmath, amssymb, amsthm, enumerate}

\title{Lecture 1}
\author{Dani Nguyen}
\date{\today}

\begin{document}
\maketitle
\tableofcontents

\section{Invertible Matrix Theorem}
\subsection{Definition}
Let A be a square n x n matrix. Then the following statements are
equivalent. That is, for a given A, the statements are either all 
true or all false.
\begin{enumerate}
    \item A is invertible.
    \item A is row equivalent to I.
    \item A has n pivot positions.
    \item The equation A\textbf{x} = \textbf{0} has only the trivial solution.
    \item The columns of A are linearly independent.
    \item The linear transformation \textbf{x} $\mapsto$ A\textbf{x} is one-to-one.
    \item The equation A\textbf{x} = \textbf{b} has at least one solution for each \textbf{b} in $\mathbb{R}^n$.
    \item The columns of A span $\mathbb{R}^n$
    \item The linear transformation \textbf{x} $\mapsto$ A\textbf{x} maps $\mathbb{R}^n$ onto $\mathbb{R}^n$.
    \item There is an n x n matrix C such that CA = I.
    \item There is an n x n matrix D such that AD = I.
    \item A$^T$ is invertible.
    \item The columns of A form a basis of $\mathbb{R}^n$.
    \item Col A = $\mathbb{R}^n$.
    \item dim Col A = n.
    \item rank A = n
    \item Nul A = {0}
    \item dim Nul A = 0
    \item The number 0 is \emph{not} an eigenvalue of A.
    \item det A $\neq$ 0.
    \item (Col A)$^\bot$ = \textbf{0}
    \item (Nul A)$^\bot$ = $\mathbb{R}^n$
    \item Row A = $\mathbb{R}^n$
\end{enumerate}

\subsection{Exercise}
\textbf{Thm 9.1.16} Let A $\in M_n$ and let $\lambda \in \mathbb{C}$. Then the following statements are equivalent:
\begin{enumerate}[(a)]
    \item $\lambda$ is an eigenvalue of A.
    \item A\textbf{x} = $\lambda$\textbf{x} for some nonzero \textbf{x} $\in \mathbb{C}^n$.
    \item (A - $\lambda$I)\textbf{x} = \textbf{0} has a nontrivial solution, that is, nullity(A - $\lambda I$) $>$ 0.
    \item rank(A - $\lambda$I) $<$ n.
    \item A - $\lambda$I is not invertible.
    \item A$^\top$ - $\lambda$I is not invertible.
    \item $\lambda$ is an eigenvalue of A$^\top$.
\end{enumerate}
\textbf{Proof}
\begin{enumerate}[(a)]
    \item $\Leftrightarrow$ (b) 
    
    By definition 9.1.1, if $A\mathbf{x}=\lambda\mathbf{x}$ and $\mathbf{x}\neq\mathbf{0}$ then $(\lambda,\mathbf{x})$
    is an eigenpair of A, meaning $\lambda$ is an eigenvalue of A.
    \item $\Leftrightarrow$ (c)
    
    These are restatements of one another.
    \item $\Leftrightarrow$ (d)
    
    By rank nullity theorem, if nullity(A - $\lambda I$) $>$ 0, then rank(A - $\lambda I$) $<$ n.
    \item $\Leftrightarrow$ (e) 
    
    By property 1 and 16 of IMT, if rank(A - $\lambda I$) $<$ n then (A - $\lambda I$) is not invertible.
    \item $\Leftrightarrow$ (f)
    
    
    \item $\Leftrightarrow$ (g)
    
    Since (a) is equivalent to (e), the same is true for $A^\top$.
\end{enumerate}

\end{document}